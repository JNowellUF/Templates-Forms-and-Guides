\documentclass[editMode]{ufdissertation}\sloppy

%%%%%%%%%%%%%%%%%%%%%%%%%%%%%%%%%%%%%%%%%%%%%%%%%%%%%%%%%%%%%%%%%%%%%%%%%%%%%%%%
%%%                 User Package and Style File loading.
%%%%%%%%%%%%%%%%%%%%%%%%%%%%%%%%%%%%%%%%%%%%%%%%%%%%%%%%%%%%%%%%%%%%%%%%%%%%%%%%

%\usepackage{CustomMacros}%  This is a user macro/style file.

\usepackage{tikz}%       tikz is used by almost everyone, but certainly by me for this.
\usepackage{pgfplots}%   pgfplots is tikz but better.
%\usepackage{amsrefs}%   amsrefs contains the .bibtex style content for mathematician papers.

\appendixGaptrue
%%%%%%%%%%%%%%%%%%%%%%%%%%%%%%%%%%%%%%%%%%%%%%%%%%%%%%%%%%%%%%%%%%%%%%%%%%%%%%%%
%%%                     User Configuration commands
%%%%%%%%%%%%%%%%%%%%%%%%%%%%%%%%%%%%%%%%%%%%%%%%%%%%%%%%%%%%%%%%%%%%%%%%%%%%%%%%

%% Uncomment the relevant line below if you have tables or figures.
\haveTablestrue%        Uncomment this if you have tables in your thesis.
\haveFigurestrue%       Uncomment this if you have figures in your thesis.
\haveObjectstrue%       Uncomment this if you have Objects in your thesis. This is almost certainly not the case however.

%%%%%%%%%%%%%%%%%%%%%%%%%%%%%%%%%%%%%%%%%%%%%%%%%%%%%%%%%%%%%%%%%%%%%%%%%%%%%%%%
%%% Below are the commands to set the degree type, department, graduation time, and chair. 
%       Most of these are self explanatory. 
%       Note: The \chair command takes an optional argument for a cochair. 
%           So if John was your chair and Jacob was a cochair, you would use \chair[Jacob]{John}.
%           If John was your chair and you had no cochair, you can simply use \chair{John}.
%%%%%%%%%%%%%%%%%%%%%%%%%%%%%%%%%%%%%%%%%%%%%%%%%%%%%%%%%%%%%%%%%%%%%%%%%%%%%%%%

\title{Dissertation and Thesis Example File}%  Put your title here.

\degreeType{Doctorate of Philosophy}%   Official name of your degree; eg "Doctorate of Philosophy".
\major{Mathematics}%                    Your official Department
\author{Jason Nowell}%                  Your Name
\thesisType{Dissertation}%              Dissertation (PhD) or Thesis (Masters)
\degreeYear{2019}%                      Intended graduation year (not the year you submit the thesis)
\degreeMonth{August}%                   Month of graduation should be May, August, or December.
\chair[Jack Griffin]{Claude Rains}%                   Chair and Cochair (see comment block above).


%%%%%%%%%%%%%%%%%%%%%%%%%%%%%%%%%%%%%%%%%%%%%%%%%%%%%%%%%%%%%%%%%%%%%%%%%%%%%%%%
%%% For each of the following, type in the name of the file that contains each section. 
%       They are assumed to be tex files, but if they aren't the command takes an optional argument for the extension.
%       So, you could load dedication.tex as your dedication file using \setDedicationFile{dedication}
%       You could load dedication.txt instead with \setDedicationFile[txt]{dedication}.
%       NOTE: For some compilers they may or may not add a .tex to the end of the file automatically.
%           If you get a "couldn't find dedication.tex.tex" type error, try the command with an empty optional argument,
%           e.g. \setDedicationFile[]{dedication}
%%%
%%%%%%%%%%%%%%%%%%%%%%%%%%%%%%%%%%%%%%%%%%%%%%%%%%%%%%%%%%%%%%%%%%%%%%%%%%%%%%%%

%%% These are REQUIRED sections; easiest to do via these commands.

\setDedicationFile{dedicationFile}%                 Dedication Page
\setAcknowledgementsFile{acknowledgementsFile}%     Acknowledgements Page
\setAbstractFile{abstractFile}%                     Abstract Page (This should only include the abstract itself)
\setReferenceFile{referenceFile}{amsplain}%         References. First argument is your bibtex source file
%                                                       the second argument is your bibtex style file.
\setBiographicalFile{biographyFile}%                Biography file of the Author (you).

%%% These are NOT required, so only use them if you actually need/have them.

%\setAbbreviationsFile{abbreviations}%           Abbreviations Page
\setAppendixFile{appendix}%                     Appendix Content; hyperlinking might be weird.
\multipleAppendixtrue%                          Uncomment this if you have more than one appendix, 
%                                                   comment it if you have only one appendix.


%%%%%%%                     End of File Assignment
%%%%%%%%%%%%%%%%%%%%%%%%%%%%%%%%%%%%%%%%%%%%%%%%%%%%%%%%%%%%%%%%%%%%%%%%%%%%%%%%

\begin{document}
%%%% Here you just need to include/input your actual work. 
%       The above files (dedication, acknowledgement, titlepage, etc etc) will all be added for you 
%       using the files you assigned above. 
%       If you want to input the above files manually you can comment out the \setFILE command above 
%       and use \input or \include here. Generally you want to use \include to get your pagebreak.
%       NOTE: If you input manually you will have to do some/all the formatting manually.



\chapter{INTRODUCTION and opening remarks} \label{intro}

We automatically capitalize all chapters, but if you need to suppress this you can use the class option ``overrideTitles" and/or ``overrideChapter" to allow you to use non-capitalized letters in the title and/or chapter names respectively. For more detailed information on the template's features and options, see the included file ``ufdissertation-Doc-and-Troubleshooting".

\renewcommand*{\thefootnote}{\fnsymbol{footnote}}\footnote{an un-numbered footnote - this is how you tell the readers that this chapter was previously published and then cite the Journal where it was published} We don't recommend that you change much of anything in the class file unless you're absolutely sure of what your are doing.\renewcommand*{\thefootnote}{\arabic{footnote}}\setcounter{footnote}{0}\footnote{and now we're back to normal footnote marking} 

\section{The Section Command Text Should Be in Title Case}

Title case is where all principal words are capitalized except prepositions, articles, and conjunctions.  %\cite{green2008wrinkle}

\subsection{Subsection Commands Are Also in Title Case}
The difference, of course, are the second level headings are left-aligned

\subsubsection{Subsubsections are in sentence case}
The third level subheadings are left-aligned but in sentence case. Only the first letter and any proper nouns are capitalized. %\cite{strickler1998contamination}

\subsubsection{If you divide a section, you must divide it into two, or more, parts}

{\bf Paragraph headings.} There is no official fourth level heading. Do not use the Paragraph heading feature in LaTeX, simply apply the bold characteristic to the first few words of a paragraph followed by a colon or period.

\subsection{I Need Another Second Level Heading in This Section}

Aliquam mi nisi, tristique at rhoncus quis, consectetur non mi. Phasellus blandit quam ligula, a viverra lacus commodo at. In iaculis nisl vel pretium sollicitudin. In efficitur massa vel elit sollicitudin, vel auctor sapien cursus. Proin feugiat sapien a mi tempus;

 $ X-X'=D+D'$

 in consequat augue cursus. Nulla sed sagittis purus. Nunc eu consequat orci, eu laoreet enim. Ut euismod tincidunt sem, eget lacinia dui luctus eu. Aliquam mi augue, faucibus id semper vitae, porta ac ligula. Morbi sed ultrices odio. Mauris id luctus ex. Nulla ac libero dictum, interdum turpis lacinia, scelerisque leo. Praesent varius orci ac eros varius pharetra.


% Modified from old template.
This is an example that uses the ``algorithm" environment to demonstrate what is happening, but it also does NOT link to any external source. To do that we need an \verb|\href| command around the actual caption text... so instead of \verb|\captionof{algorithm}{Test Caption}| it would be something like \verb|\caption{algorithm}{www.google.com}{\href{Test Caption}}|.
\begin{algorithm}% Example showing the weird "algorithm" environment works...
    \captionof{algorithm}{Test Caption}
\end{algorithm}
\addObject{TestStuff!}{www.google.com}%     This is probably the command that a normal author will use to add objects.

\chapter{LITERATURE REVIEW} \label{lit}

\section{Dolor Sit Amet}

 Many of the problems in theses and dissertations involve tables. The UF Graduate Counsel is very specific in the Table Requirements.There should be no vertical lines in tables and only three horizontal lines. No bold text, etc., see the web site for the complete list of requirements. One simple improvement can be incorporated by using tabularx instead of the tabular environment. This allows a table to be stretched the full text width easily, which avoids the centered or left aligned issue. Table \ref{first} is an examble of the tabularx code. Consectetur adipiscing elit. Fusce eget tempus lectus, non porttitor tellus. Aliquam molestie sed urna quis convallis. Aenean nibh eros, aliquam non eros in, tempus lacinia justo. In magna sapien, blandit a faucibus ac, scelerisque nec purus. 
 
\begin{table}[htbp]% Fix the table captions to sit directly on the table, but figures do NOT sit directly on the figure.
    \captionof{table}{A sample Table using tabularx}\label{first}
    \begin{tabularx}{6.5in}{XXX}
      \hline
      First & Second & Third \\
      \hline
      12 & 45 & 26 \\
      17 & 32 & 93 \\
      text & 51 & can be there too. \\	
      \hline
    \end{tabularx}
\end{table}
 
 
 Praesent fermentum felis nec massa interdum, vel dapibus mi luctus. Cras id fringilla mauris. Ut molestie eros mi, ut hendrerit nulla tempor et. Pellentesque tortor quam, mattis a scelerisque nec, euismod et odio. Mauris rhoncus metus sit amet risus mattis, eu mattis sem interdum.

 \begin{table}[htbp]
    \caption{A sample Table using standard tablular}\label{first}
    \begin{tabular}{c c c}
      \hline
      First & Second & Third \\
      \hline
      12 & 45 & 26 \\
      17 & 32 & 93 \\
      text & 51 & can be there too. \\	
      \hline
    \end{tabular}
\end{table}

\subsection{Platea Dictumst}
Donec convallis scelerisque ante, in sollicitudin orci laoreet eu. Nam arcu magna, semper vel lorem eu, venenatis ultrices est. Nam aliquet ut erat ac scelerisque. Maecenas ut molestie mi. Phasellus ipsum magna, sollicitudin eu ipsum quis, imperdiet cursus turpis. Etiam pretium enim a fermentum accumsan. Morbi vel vehicula enim.



\section{Ex id ullamcorper commodo}
Augue sapien mattis leo, nec accumsan turpis quam at neque. Ut pellentesque velit sed placerat cursus. Integer congue urna non massa dictum, a pellentesque arcu accumsan. Nulla posuere, elit accumsan eleifend elementum, ipsum massa tristique metus, in ornare neque nisl sed odio. Nullam eget elementum nisi. Duis a consectetur erat, sit amet malesuada sapien. Aliquam nec sapien et leo sagittis porttitor at ut lacus. Vivamus vulputate elit vitae libero condimentum dictum. Nulla facilisi. Quisque non nibh et massa ullamcorper iaculis.

Integer laoreet bibendum arcu non pulvinar. Curabitur ac magna nibh. Phasellus sed nisi semper, molestie neque at, tempus lacus. Aenean vitae lacinia est. Phasellus aliquam lacus sit amet placerat molestie. Sed sit amet bibendum lectus, ac ornare ligula. Curabitur porttitor interdum tortor a dignissim. Quisque a placerat nibh. Phasellus lobortis imperdiet augue, non congue est bibendum eu. Vivamus tincidunt quam eu fringilla laoreet.

Maecenas efficitur dolor et ipsum convallis, ut fringilla neque luctus. Donec ac nisl quis leo gravida accumsan sit amet sed tellus. Quisque placerat hendrerit augue sit amet aliquet. Vestibulum laoreet consequat nunc, et egestas nisl auctor et. Duis scelerisque vulputate placerat. Proin tempus ligula ac tempor eleifend. Nullam est odio, commodo quis nisl eu, feugiat efficitur purus.

Duis egestas in mauris vel efficitur. Sed a faucibus sem, non euismod enim. Maecenas nec nulla justo. Suspendisse ut orci ac mi aliquet tincidunt ac eget quam. Quisque ac mi sagittis, dapibus dui a, facilisis neque. Aenean euismod orci sem, non imperdiet ipsum pulvinar ac. Proin eu vestibulum magna, eu ullamcorper nulla. Etiam enim felis, dignissim eget commodo ac, faucibus nec justo. Nulla condimentum velit imperdiet ligula aliquam semper. Nulla facilisi. Ut in lobortis metus, at dictum ipsum. Suspendisse facilisis nec eros eget mollis. Vestibulum eget dolor ac mauris lobortis gravida. Suspendisse consectetur orci in risus pharetra, sed eleifend nisl lacinia. Mauris augue nibh, commodo sed sem at, congue molestie massa. Suspendisse sodales aliquet tellus, a tristique nunc aliquam id.

% Modified from old template.
\chapter{MATERIALS ANS METHODS} \label{materials}

\section{Consectetur Adipiscing Elit}

 Fusce eget tempus lectus, non porttitor tellus. Aliquam molestie sed urna quis convallis. Aenean nibh eros, aliquam non eros in, tempus lacinia justo. In magna sapien, blandit a faucibus ac, scelerisque nec purus. Praesent fermentum felis nec massa interdum, vel dapibus mi luctus. Cras id fringilla mauris. Ut molestie eros mi, ut hendrerit nulla tempor et. Pellentesque tortor quam, mattis a scelerisque nec, euismod et odio. Mauris rhoncus metus sit amet risus mattis, eu mattis sem interdum.

\subsection{This Is an Isolated Heading}
Either promote this to a section heading, add another subsection heading, or delete this heading. A random citation to demonstrate the bibliography; \cite{Rudin-UnitBallCN}

\section{Augue sapien mattis leo}
Nec accumsan turpis quam at neque. Ut pellentesque velit sed placerat cursus. Integer congue urna non massa dictum, a pellentesque arcu accumsan. Nulla posuere, elit accumsan eleifend elementum, ipsum massa tristique metus, in ornare neque nisl sed odio. Nullam eget elementum nisi. Duis a consectetur erat, sit amet malesuada sapien. Aliquam nec sapien et leo sagittis porttitor at ut lacus. Vivamus vulputate elit vitae libero condimentum dictum. Nulla facilisi. Quisque non nibh et massa ullamcorper iaculis. A random citation to demonstrate the bibliography; \cite{ConwayCSV1}

% Modified from old template.

\chapter{EXAMPLES OF EDITOR/Author TOOLS, TABLES, AND IMAGES}% Notice that we can use chapter/section etc breaks in the master file if we want, and then use \input instead of \include to avoid unneccessary page breaks.
\section{Example of using the authorRemark and editorRemark}
If you don't see any blue or red type under this line, then you almost certainly need to include the optional ``editMode" to the document class. Thus your document class (first line) should read \verb|\documentclass[editMode]{ufdissertation}|.

\authorRemark{Test! This is a remark written by the author, to themselves, for review purposes. It will be suppressed unless editMode is used in the class options.}

\editorRemark{This is an editor's remark, written by an editor in-line so that they can write into the content itself with something easy to see. But the remark will be suppressed unless editMode is used in the class options.

To get this remark to go away, simply remove ``editMode" from the documentclass options at the top of the user's tex-file. This also removes the blue Author Remarks.
}%     Stuff about using editorRemark and authorRemark commands
\section{Table Examples}% Notice that the section command needs to be included in the file somewhere. The \include command will not generate chapter or section breaks automatically.
You may notice that some tables get moved outside of where you placed them. This is because \LaTeX{} is a little too helpful when it comes to placement of `float' types; which includes tables and figures. You can get around this by using the ``H" parameter in the table environment, or the `multiFigure' environment described in the ``adding graphics section"; ie section \ref{Sec:addingGraphics}

\begin{table}[H]
\begin{tabular}{llcr}
Some    & Data  & Goes  & Here\\
Some    & Data  & Goes  & Here\\
Some    & Data  & Goes  & Here\\
Some    & Data  & Goes  & Here\\
\end{tabular}
\caption[An example of a table caption in the incorrect place.]{This table is located in the correct section because it uses the ``H" optional parameter in the table environment, unlike the next tables which have been helpfully moved by \LaTeX{} to the next page, which places them inside the section.

You should also make a note that the caption command is placed after the table itself, which means the caption occurs after the table. The graduate school requires tables to have captions placed {before} the actual table data, so the caption command should be located before the table data. See the next table for an example.}
\end{table}

\begin{table}[]
\caption[A proper table caption location]{Notice that this caption is included above the table data, as per the graduate school requirements. Also note that the caption itself has a short version in the ``List of Tables" which is achieved by using the optional argument of the caption command. See the file source code directly to see the example.

Unfortunately, since we did not use the ``H" parameter in the table environment, this table was placed \textit{after} the next section heading, which is almost certainly not where an author would have wanted it.}
%\begin{center}
\begin{tabularx}{\textwidth}{XXXX}\hline
Some    & Data  & Goes  & Here\\\hline
Some    & Data  & Goes  & Here\\
Some    & Data  & Goes  & Here\\
Some    & Data  & Goes  & Here\\\hline
\end{tabularx}
%\end{center}

\end{table}

\section{Very Long Tables}

There are two approaches to inputting very long tables. You can do it manually, or you can do it using the longtables package. Here we include an example of both. Table \ref{tbl1} is done manually, whereas \ref{tbl2} is done using the longtables package.

\begin{table}[H]
\caption{Feasible triples for highly variable Grid, MLMMH.} \label{tbl1}
\begin{tabularx}{6.5 in}{r l X}
\hline {{Time (s)}} & {{Triple chosen}} & {{Other feasible triples}} \\ \hline
0 & (1, 11, 13725) & (1, 12, 10980), (1, 13, 8235), (2, 2, 0), (3, 1, 0) \\
2745 & (1, 12, 10980) & (1, 13, 8235), (2, 2, 0), (2, 3, 0), (3, 1, 0) \\
5490 & (1, 12, 13725) & (2, 2, 2745), (2, 3, 0), (3, 1, 0) \\
8235 & (1, 12, 16470) & (1, 13, 13725), (2, 2, 2745), (2, 3, 0), (3, 1, 0) \\
10980 & (1, 12, 16470) & (1, 13, 13725), (2, 2, 2745), (2, 3, 0), (3, 1, 0) \\
13725 & (1, 12, 16470) & (1, 13, 13725), (2, 2, 2745), (2, 3, 0), (3, 1, 0) \\
16470 & (1, 13, 16470) & (2, 2, 2745), (2, 3, 0), (3, 1, 0) \\
19215 & (1, 12, 16470) & (1, 13, 13725), (2, 2, 2745), (2, 3, 0), (3, 1, 0) \\
21960 & (1, 12, 16470) & (1, 13, 13725), (2, 2, 2745), (2, 3, 0), (3, 1, 0) \\
24705 & (1, 12, 16470) & (1, 13, 13725), (2, 2, 2745), (2, 3, 0), (3, 1, 0) \\
27450 & (1, 12, 16470) & (1, 13, 13725), (2, 2, 2745), (2, 3, 0), (3, 1, 0) \\
30195 & (2, 2, 2745) & (2, 3, 0), (3, 1, 0) \\
32940 & (1, 13, 16470) & (2, 2, 2745), (2, 3, 0), (3, 1, 0) \\
35685 & (1, 13, 13725) & (2, 2, 2745), (2, 3, 0), (3, 1, 0) \\
38430 & (1, 13, 10980) & (2, 2, 2745), (2, 3, 0), (3, 1, 0) \\
41175 & (1, 12, 13725) & (1, 13, 10980), (2, 2, 2745), (2, 3, 0), (3, 1, 0) \\
43920 & (1, 13, 10980) & (2, 2, 2745), (2, 3, 0), (3, 1, 0) \\
46665 & (2, 2, 2745) & (2, 3, 0), (3, 1, 0) \\
49410 & (2, 2, 2745) & (2, 3, 0), (3, 1, 0) \\
52155 & (1, 12, 16470) & (1, 13, 13725), (2, 2, 2745), (2, 3, 0), (3, 1, 0) \\
54900 & (1, 13, 13725) & (2, 2, 2745), (2, 3, 0), (3, 1, 0) \\
57645 & (1, 13, 13725) & (2, 2, 2745), (2, 3, 0), (3, 1, 0) \\
60390 & (1, 12, 13725) & (2, 2, 2745), (2, 3, 0), (3, 1, 0) \\
63135 & (1, 13, 16470) & (2, 2, 2745), (2, 3, 0), (3, 1, 0) \\
65880 & (1, 13, 16470) & (2, 2, 2745), (2, 3, 0), (3, 1, 0) \\
68625 & (2, 2, 2745) & (2, 3, 0), (3, 1, 0) \\
71370 & (1, 13, 13725) & (2, 2, 2745), (2, 3, 0), (3, 1, 0) \\
74115 & (1, 12, 13725) & (2, 2, 2745), (2, 3, 0), (3, 1, 0) \\
76860 & (1, 13, 13725) & (2, 2, 2745), (2, 3, 0), (3, 1, 0) \\
79605 & (1, 13, 13725) & (2, 2, 2745), (2, 3, 0), (3, 1, 0) \\
82350 & (1, 12, 13725) & (2, 2, 2745), (2, 3, 0), (3, 1, 0) \\
\hline
\end{tabularx}
\end{table}

\begin{table}[h!t!]
\begin{tabularx}{6.5 in}{r l X}
\multicolumn{3}{l}{Table \ref{tbl1}. Continued}\\%
\hline {{Time (s)}} & {{Triple chosen}} & {{Other feasible triples}} \\ \hline
85095 & (1, 12, 13725) & (1, 13, 10980), (2, 2, 2745), (2, 3, 0), (3, 1, 0) \\
87840 & (1, 13, 16470) & (2, 2, 2745), (2, 3, 0), (3, 1, 0) \\
90585 & (1, 13, 16470) & (2, 2, 2745), (2, 3, 0), (3, 1, 0) \\
93330 & (1, 13, 13725) & (2, 2, 2745), (2, 3, 0), (3, 1, 0) \\
96075 & (1, 13, 16470) & (2, 2, 2745), (2, 3, 0), (3, 1, 0) \\
98820 & (1, 13, 16470) & (2, 2, 2745), (2, 3, 0), (3, 1, 0) \\
101565 & (1, 13, 13725) & (2, 2, 2745), (2, 3, 0), (3, 1, 0) \\
104310 & (1, 13, 16470) & (2, 2, 2745), (2, 3, 0), (3, 1, 0) \\
107055 & (1, 13, 13725) & (2, 2, 2745), (2, 3, 0), (3, 1, 0) \\
109800 & (1, 13, 13725) & (2, 2, 2745), (2, 3, 0), (3, 1, 0) \\
112545 & (1, 12, 16470) & (1, 13, 13725), (2, 2, 2745), (2, 3, 0), (3, 1, 0) \\
115290 & (1, 13, 16470) & (2, 2, 2745), (2, 3, 0), (3, 1, 0) \\
118035 & (1, 13, 13725) & (2, 2, 2745), (2, 3, 0), (3, 1, 0) \\
120780 & (1, 13, 16470) & (2, 2, 2745), (2, 3, 0), (3, 1, 0) \\
123525 & (1, 13, 13725) & (2, 2, 2745), (2, 3, 0), (3, 1, 0) \\
126270 & (1, 12, 16470) & (1, 13, 13725), (2, 2, 2745), (2, 3, 0), (3, 1, 0) \\
129015 & (2, 2, 2745) & (2, 3, 0), (3, 1, 0) \\
131760 & (2, 2, 2745) & (2, 3, 0), (3, 1, 0) \\
134505 & (1, 13, 16470) & (2, 2, 2745), (2, 3, 0), (3, 1, 0) \\
137250 & (1, 13, 13725) & (2, 2, 2745), (2, 3, 0), (3, 1, 0) \\
139995 & (2, 2, 2745) & (2, 3, 0), (3, 1, 0) \\
142740 & (2, 2, 2745) & (2, 3, 0), (3, 1, 0) \\
145485 & (1, 12, 16470) & (1, 13, 13725), (2, 2, 2745), (2, 3, 0), (3, 1, 0)\\%
148230 & (2, 2, 2745) & (2, 3, 0), (3, 1, 0) \\
150975 & (1, 13, 16470) & (2, 2, 2745), (2, 3, 0), (3, 1, 0) \\
153720 & (1, 12, 13725) & (2, 2, 2745), (2, 3, 0), (3, 1, 0) \\
156465 & (1, 13, 13725) & (2, 2, 2745), (2, 3, 0), (3, 1, 0) \\
159210 & (1, 13, 13725) & (2, 2, 2745), (2, 3, 0), (3, 1, 0) \\
161955 & (1, 13, 16470) & (2, 2, 2745), (2, 3, 0), (3, 1, 0) \\
164700 & (1, 13, 13725) & (2, 2, 2745), (2, 3, 0), (3, 1, 0) \\
\hline
\end{tabularx}
\end{table}
\newpage

Alternatively, compared to the previous example where we used manual breaks to break the table, we can let LaTeX do this for us, as well as taking care of any recurrent headers and footers, utilizing the \verb|\longtable| command,\footnote{note that the longtable environment is not in a table environment; putting it inside a table environment will stop it from correctly page breaking as needed.} as follows:


\begin{longtable}[h!t!]{p{0.6in}p{1in}p{4.4in}}
    \caption{Duplicate of Previous table, using longtables environment.}\label{tbl2}\\% Default caption at top of table
    \hline {{Time (s)}} & {{Triple chosen}} & {{Other feasible triples}}\\ \hline \endfirsthead% The top row of the first page
    \hline\endfoot%         This line should always be included; it includes a line at the end of the table on every page.
    \caption*{continued} \\% This caption is added to every page after the first as per the \endhead next line.
    \hline{{Time (s)}} & {{Triple chosen}} & {{Other feasible triples}}\\ \hline \endhead%
%                                                               Everything between \endfoot and \endhead here is added at
%                                                                   the top of the table on every page except the first;
%                                                                   The first page is an exception because we have defined a
%                                                                   \endfirsthead row already which superceeds \endhead.
0 & (1, 11, 13725) & (1, 12, 10980), (1, 13, 8235), (2, 2, 0), (3, 1, 0)\\
2745 & (1, 12, 10980) & (1, 13, 8235), (2, 2, 0), (2, 3, 0), (3, 1, 0) \\
5490 & (1, 12, 13725) & (2, 2, 2745), (2, 3, 0), (3, 1, 0) \\
8235 & (1, 12, 16470) & (1, 13, 13725), (2, 2, 2745), (2, 3, 0), (3, 1, 0) \\
10980 & (1, 12, 16470) & (1, 13, 13725), (2, 2, 2745), (2, 3, 0), (3, 1, 0) \\
13725 & (1, 12, 16470) & (1, 13, 13725), (2, 2, 2745), (2, 3, 0), (3, 1, 0) \\
16470 & (1, 13, 16470) & (2, 2, 2745), (2, 3, 0), (3, 1, 0) \\
19215 & (1, 12, 16470) & (1, 13, 13725), (2, 2, 2745), (2, 3, 0), (3, 1, 0) \\
21960 & (1, 12, 16470) & (1, 13, 13725), (2, 2, 2745), (2, 3, 0), (3, 1, 0) \\
24705 & (1, 12, 16470) & (1, 13, 13725), (2, 2, 2745), (2, 3, 0), (3, 1, 0) \\
27450 & (1, 12, 16470) & (1, 13, 13725), (2, 2, 2745), (2, 3, 0), (3, 1, 0) \\
30195 & (2, 2, 2745) & (2, 3, 0), (3, 1, 0) \\
32940 & (1, 13, 16470) & (2, 2, 2745), (2, 3, 0), (3, 1, 0) \\
35685 & (1, 13, 13725) & (2, 2, 2745), (2, 3, 0), (3, 1, 0) \\
38430 & (1, 13, 10980) & (2, 2, 2745), (2, 3, 0), (3, 1, 0) \\
41175 & (1, 12, 13725) & (1, 13, 10980), (2, 2, 2745), (2, 3, 0), (3, 1, 0) \\
43920 & (1, 13, 10980) & (2, 2, 2745), (2, 3, 0), (3, 1, 0) \\
46665 & (2, 2, 2745) & (2, 3, 0), (3, 1, 0) \\
49410 & (2, 2, 2745) & (2, 3, 0), (3, 1, 0) \\
52155 & (1, 12, 16470) & (1, 13, 13725), (2, 2, 2745), (2, 3, 0), (3, 1, 0) \\
54900 & (1, 13, 13725) & (2, 2, 2745), (2, 3, 0), (3, 1, 0) \\
57645 & (1, 13, 13725) & (2, 2, 2745), (2, 3, 0), (3, 1, 0) \\
60390 & (1, 12, 13725) & (2, 2, 2745), (2, 3, 0), (3, 1, 0) \\
63135 & (1, 13, 16470) & (2, 2, 2745), (2, 3, 0), (3, 1, 0) \\
65880 & (1, 13, 16470) & (2, 2, 2745), (2, 3, 0), (3, 1, 0) \\
68625 & (2, 2, 2745) & (2, 3, 0), (3, 1, 0) \\
71370 & (1, 13, 13725) & (2, 2, 2745), (2, 3, 0), (3, 1, 0) \\
74115 & (1, 12, 13725) & (2, 2, 2745), (2, 3, 0), (3, 1, 0) \\
76860 & (1, 13, 13725) & (2, 2, 2745), (2, 3, 0), (3, 1, 0) \\
79605 & (1, 13, 13725) & (2, 2, 2745), (2, 3, 0), (3, 1, 0) \\
82350 & (1, 12, 13725) & (2, 2, 2745), (2, 3, 0), (3, 1, 0) \\
85095 & (1, 12, 13725) & (1, 13, 10980), (2, 2, 2745), (2, 3, 0), (3, 1, 0) \\
87840 & (1, 13, 16470) & (2, 2, 2745), (2, 3, 0), (3, 1, 0) \\
90585 & (1, 13, 16470) & (2, 2, 2745), (2, 3, 0), (3, 1, 0) \\
93330 & (1, 13, 13725) & (2, 2, 2745), (2, 3, 0), (3, 1, 0) \\
96075 & (1, 13, 16470) & (2, 2, 2745), (2, 3, 0), (3, 1, 0) \\
98820 & (1, 13, 16470) & (2, 2, 2745), (2, 3, 0), (3, 1, 0) \\
101565 & (1, 13, 13725) & (2, 2, 2745), (2, 3, 0), (3, 1, 0) \\
104310 & (1, 13, 16470) & (2, 2, 2745), (2, 3, 0), (3, 1, 0) \\
107055 & (1, 13, 13725) & (2, 2, 2745), (2, 3, 0), (3, 1, 0) \\
109800 & (1, 13, 13725) & (2, 2, 2745), (2, 3, 0), (3, 1, 0) \\
112545 & (1, 12, 16470) & (1, 13, 13725), (2, 2, 2745), (2, 3, 0), (3, 1, 0) \\
115290 & (1, 13, 16470) & (2, 2, 2745), (2, 3, 0), (3, 1, 0) \\
118035 & (1, 13, 13725) & (2, 2, 2745), (2, 3, 0), (3, 1, 0) \\
120780 & (1, 13, 16470) & (2, 2, 2745), (2, 3, 0), (3, 1, 0) \\
123525 & (1, 13, 13725) & (2, 2, 2745), (2, 3, 0), (3, 1, 0) \\
126270 & (1, 12, 16470) & (1, 13, 13725), (2, 2, 2745), (2, 3, 0), (3, 1, 0) \\
129015 & (2, 2, 2745) & (2, 3, 0), (3, 1, 0) \\
131760 & (2, 2, 2745) & (2, 3, 0), (3, 1, 0) \\
134505 & (1, 13, 16470) & (2, 2, 2745), (2, 3, 0), (3, 1, 0) \\
137250 & (1, 13, 13725) & (2, 2, 2745), (2, 3, 0), (3, 1, 0) \\
139995 & (2, 2, 2745) & (2, 3, 0), (3, 1, 0) \\
142740 & (2, 2, 2745) & (2, 3, 0), (3, 1, 0) \\
145485 & (1, 12, 16470) & (1, 13, 13725), (2, 2, 2745), (2, 3, 0), (3, 1, 0)\\%
148230 & (2, 2, 2745) & (2, 3, 0), (3, 1, 0) \\
150975 & (1, 13, 16470) & (2, 2, 2745), (2, 3, 0), (3, 1, 0) \\
153720 & (1, 12, 13725) & (2, 2, 2745), (2, 3, 0), (3, 1, 0) \\
\newpage% Force a pagebreak here so that we don't have a stranded row.
156465 & (1, 13, 13725) & (2, 2, 2745), (2, 3, 0), (3, 1, 0) \\
159210 & (1, 13, 13725) & (2, 2, 2745), (2, 3, 0), (3, 1, 0) \\
161955 & (1, 13, 16470) & (2, 2, 2745), (2, 3, 0), (3, 1, 0) \\
164700 & (1, 13, 13725) & (2, 2, 2745), (2, 3, 0), (3, 1, 0) \\
\end{longtable}

%    Stuff about using Tables.
\section{Examples of Adding Graphics}
\label{Sec:addingGraphics}
All of the below code with subfigures A-Z was generated with:
\begin{verbatim}
\begin{multiFigure}
\addFigure{0.3}{./theworld.png}
\addFigure{0.2}{./theworld.png}
\addFigure{0.4}{./theworld.png}
\addFigure[Z]{0.6}{./theworld.png}
\captionof{figure}[This is a test caption.]{This is a test caption. 
This text has the bit for the whole figure. 
Meanwhile, subfigure A is weird looking map. 
Subfigure B is a smaller map. 
And Subfigure C is a bigger but still weird looking map. 
Moreover, I can override the map, which is why Z is 
another weird map that came after map C.}
\end{multiFigure}
\end{verbatim}
Note that \LaTeX{} can be pretty fickle when it comes to placing figures relative to text near the figure. Specifically, the ``Figure" environment is a `float' type, which is placed somewhere ``nearby" where it appears in the text, which can be pretty frustrating. For this reason I have circumvented the `float' part of the figure in order to allow more control over the figure placement. So if one uses the \verb|\begin{figure}\end{figure}| construction, the figure may appear in a slightly weird place, whereas you can use the \verb|\begin{multiFigure}\end{multiFigure}| even with only 1 figure, to force placement to work. The only caveat here is that captions need to be placed using the command \verb|\captionof{<NAME>}[<LIST-ENTRY>]{<CAPTION>}| where NAME is the type of caption, LIST-ENTRY is what appears in the `List of' at the beginning of the thesis, and CAPTION is the actual caption.

\begin{multiFigure}
\addFigure{0.3}{./theworld.png}
\addFigure{0.2}{./theworld.png}
\addFigure{0.4}{./theworld.png}
\addFigure[Z]{0.6}{./theworld.png}
\captionof{figure}[This is a test caption.]{This is a test caption. This text has the bit for the whole figure. Meanwhile, subfigure A is weird looking map. Subfigure B is a smaller map. And Subfigure C is a bigger but still weird looking map. Moreover, I can override the map, which is why Z is another weird map that came after map C.}
\end{multiFigure}

\begin{multiFigure}
\addFigure{0.9}{./theworld.png}
\captionof{figure}{This is a super-long caption to make sure that the caption in the list-of section is correctly single space with the blank white line between captions. That being said, you should probably always use the list-entry optional argument in the captionof command to write a shorter caption instead of this nonsense.}
\end{multiFigure}

\section{A Note On Graphics}
The command \verb|\addFigure| in the multiFigure environment, and/or the command \verb|\includegraphics| will take almost every type of graphic file currently in use as of the writing of this template. The only notable exception is the bitmap, ie .bmp file. Most software won't save to bitmap without specifically requesting it at this point, but if you have generated a .bmp file you can load it in most any graphic editor (eg MSpaint or photoshop) and save it as a different file type, such as .PNG which is significantly smaller file size as well. Note that the commands typically require the file extension to be included, and it is case sensitive. Thus in the above \verb|\addFigure{0.2}{./theworld.png}| works but \verb|\addFigure{0.2}{./theworld.PNG}| would error and \verb|\addFigure{0.2}{./theworld}| may or may not work depending on which specific TeX editor you are using.

%    Stuff about using Images.

\chapter{SUMMARY AND CONCLUSIONS} \label{conclusion}

\section{Non Porttitor Tellus}

Aliquam molestie sed urna quis convallis. Aenean nibh eros, aliquam non eros in, tempus lacinia justo. In magna sapien, blandit a faucibus ac, scelerisque nec purus. Praesent fermentum felis nec massa interdum, vel dapibus mi luctus. Cras id fringilla mauris. Ut molestie eros mi, ut hendrerit nulla tempor et. Pellentesque tortor quam, mattis a scelerisque nec, euismod et odio. Mauris rhoncus metus sit amet risus mattis, eu mattis sem interdum.

\subsection{Nam Arcu Magna}
Semper vel lorem eu, venenatis ultrices est. Nam aliquet ut erat ac scelerisque. Maecenas ut molestie mi. Phasellus ipsum magna, sollicitudin eu ipsum quis, imperdiet cursus turpis. Etiam pretium enim a fermentum accumsan. Morbi vel vehicula enim.

\subsubsection{Ut pellentesque velit sede}
 Placerat cursus. Integer congue urna non massa dictum, a pellentesque arcu accumsan. Nulla posuere, elit accumsan eleifend elementum, ipsum massa tristique metus, in ornare neque nisl sed odio. Nullam eget elementum nisi. Duis a consectetur erat, sit amet malesuada sapien. Aliquam nec sapien et leo sagittis porttitor at ut lacus. Vivamus vulputate elit vitae libero condimentum dictum. Nulla facilisi. Quisque non nibh et massa ullamcorper iaculis.

% Modified from old template.



\end{document}

